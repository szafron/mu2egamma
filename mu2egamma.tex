%% LyX 2.1.0 created this file.  For more info, see http://www.lyx.org/.
%% Do not edit unless you really know what you are doing.
\documentclass[english]{article}
\usepackage[T1]{fontenc}
\usepackage[latin9]{inputenc}
\usepackage{geometry}
\geometry{verbose,tmargin=1cm,bmargin=1cm,lmargin=1cm,rmargin=1cm,headheight=1cm,headsep=1cm,footskip=1cm}
\setlength{\parskip}{\smallskipamount}
\setlength{\parindent}{0pt}
\usepackage{float}
\usepackage{amstext}

\makeatletter

%%%%%%%%%%%%%%%%%%%%%%%%%%%%%% LyX specific LaTeX commands.
%% Because html converters don't know tabularnewline
\providecommand{\tabularnewline}{\\}

%%%%%%%%%%%%%%%%%%%%%%%%%%%%%% User specified LaTeX commands.
\renewcommand\[{\begin{equation}}
\renewcommand\]{\end{equation}}
\usepackage{slashed}
\usepackage{braket}

\makeatother

\usepackage{babel}
\begin{document}

\section{$\mu\rightarrow e\gamma$ in LR symmetric model}

We consider diagrams with $W_{R}$ and $W_{L}$ exchange. We include
mixing between L-R W bosns, mixing between heavy-light neutrinos.
We paramterize charged current interaction such that

\begin{eqnarray}
\mathcal{L}_{W} & = & \frac{g}{\sqrt{2}}\overline{N}\left(K_{R}^{R}\gamma^{\mu}P_{R}+K_{L}^{R}\gamma^{\mu}P_{L}\right)lW_{R\mu}^{+}\\
 & + & \frac{g}{\sqrt{2}}\overline{N}\left(K_{R}^{L}\gamma^{\mu}P_{R}+K_{L}^{L}\gamma^{\mu}P_{L}\right)lW_{L\mu}^{+}
\end{eqnarray}


Typically $K_{R}^{R}$ and $K_{L}^{L}$ are large, while $K_{L}^{R}$
and $K_{R}^{L}$ are suppressed by L-R mixing. 

Branching ratio for $\mu\rightarrow e\gamma$ is \cite{Bu:2008fx}

\begin{eqnarray*}
Br(\mu & \rightarrow & e\gamma)=\frac{3\alpha}{8\pi}\left|\sum_{h=L,R}\left(\frac{m_{W_{L}}}{m_{W_{h}}}\right)^{2}\sum_{i}\left[\left(K_{R}^{h\dagger}\right)_{ei}\left(K_{R}^{h}\right)_{i\mu}F\left(\frac{m_{i}^{2}}{m_{W_{h}}^{2}}\right)+\left(\frac{m_{i}}{m_{\mu}}\right)\left(K_{L}^{h\dagger}\right)_{ei}\left(K_{R}^{h}\right)_{i\mu}G\left(\frac{m_{i}^{2}}{m_{W_{h}}^{2}}\right)\right]\right|^{2},\\
 &  & +\frac{3\alpha}{8\pi}\left|\left(\frac{m_{W}}{m_{W_{R}}}\right)^{2}\sum_{i}\left[\left(K_{L}^{h\dagger}\right)_{ei}\left(K_{L}^{h}\right)_{i\mu}F\left(\frac{m_{i}^{2}}{m_{W_{h}}^{2}}\right)+\left(\frac{m_{i}}{m_{\mu}}\right)\left(K_{R}^{h\dagger}\right)_{ei}\left(K_{L}^{h}\right)_{i\mu}G\left(\frac{m_{i}^{2}}{m_{W_{h}}^{2}}\right)\right]\right|^{2}\\
\end{eqnarray*}
where $m_{i}$ is the mass of the heavy neutrino. Function $F$ is
given by:

\[
F(x)=\frac{1}{6\left(1-x\right)^{4}}\left[10-43x+78x^{2}-49x^{3}+4x^{4}+18x^{3}\ln x\right].
\]
And function $G$ is 
\[
G(x)=\frac{1}{\left(1-x\right)^{3}}\left[-4+15x-12x^{2}+x^{3}+6x^{2}\ln x\right].
\]
Sum over $i$ runs over all six neutrinos. Light neutrino contribution
is negligible, unless there is a large contribution due to the non
unitarity of $K_{L}$. 

For unitary $K_{R}$, ($\left(K_{R}^{\dagger}K_{R}\right)_{e\mu}=0$)
and for $\left(\frac{m_{i}}{m_{W_{R}}}\right)\ll1$, assuming mixing
only between two generations we get simplified equation
\begin{equation}
Br(\mu\rightarrow e\gamma)=\frac{3\alpha}{32\pi}\left(\frac{m_{W}}{m_{W_{R}}}\right)^{4}\left(\sin\theta\mbox{\ensuremath{\cos\theta}}\frac{\Delta m_{12}^{2}}{m_{W_{R}}^{2}}\right)^{2}.\label{eq:mu2egamma}
\end{equation}
Notice that the same equation holds also in the case of light neutrinos,
if we put $m_{W_{R}}=m_{W}$. This contribution is negligible since
\[
\frac{\Delta m_{12}^{2}}{m_{W}^{2}}\approx\frac{10^{-3}\text{{eV}}^{2}}{80\text{{GeV}}^{2}}\sim10^{-25}.
\]


For $M_{W_{R}}=2200$GeV the branching ratio is suppressed by $\frac{3\alpha}{8\pi}\left(\frac{m_{W}}{m_{W_{R}}}\right)^{4}\sim10^{-9}$,
which gives good estimation of the order of magnitude. To get an agreement
with experimental bounds we need to adjust masses and mixing, such
that $\left|\left[K_{R}^{\dagger}F\left(\frac{m_{i}^{2}}{m_{W_{R}}^{2}}\right)K_{R}\right]_{e\mu}\right|^{2}\ll1$.
For unitary $K_{R}$ the important suppression comes, not directly
from the masses of neutrinos, but rather from the splitting of the
masses. For maximal mixing between two generations, $M_{W_{R}}=2200GeV$
and neutrinos masses at the scale of TeV, we need the splitting to
be of the order of about 100GeV to get an agreement with the current
bounds. 

Based on \cite{Mohapatra:1992uu} we get contribution form scalars
$\Delta_{L,R}$
\[
Br\left(\mu\rightarrow e\gamma\right)=\frac{\alpha}{48\pi G_{F}^{2}}\left[\frac{\left(f'^{\dagger}f'\right)_{\mu e}}{M_{\Delta^{++}}^{2}}+\frac{\left(f{}^{\dagger}f\right)_{\mu e}}{M_{\Delta^{+}}^{2}}\right]^{2}
\]
where we have assumed following interaction
\[
\mathcal{L}_{\Delta}=l_{L}^{T}C^{-1}fl_{L}\Delta_{L}^{++}+l_{R}^{T}C^{-1}fl_{R}\Delta_{R}^{++}+\frac{1}{\sqrt{2}}\left(l_{L}^{T}C^{-1}fK_{L}^{T}v_{L}+v_{L}^{T}K_{L}fC^{-1}l_{L}\right)\Delta_{L}^{+}.
\]
Also $f'=2f$ for vertices with two identical particles. We usually
denote $\Delta_{L}^{+}=H_{1}^{+}.$ Contribution form the $H_{2}^{+}$
is given by
\[
Br\left(\mu\rightarrow e\gamma\right)=\frac{3\alpha}{4\pi}\frac{\sin^{2}2\beta}{G_{F}^{2}\kappa^{2}M_{H_{2}^{+}}^{4}m_{\mu}^{2}}\left|\left(m_{l}f^{T}K_{R}^{T}M_{N}K_{R}\right)_{\mu e}\right|^{2}
\]
where $\beta=\frac{\kappa}{v_{R}}$. In a table we have summarized
current and planned limits on various charged lepton flavor violating
processes. 

Currently we need to focus on $\mu\rightarrow e\gamma$, which gives
the best constrains. When we adjust the mixing, we also need to remember
about $\tau$ CLFV decays. Other process will be important in future,
but they also require significantly more work, since the direct contributions
have to be evaluated. They usually require calculation of box diagrams.
The best solution here would be to see if we can somehow combine our
FeynRules implementation of LR model, with \cite{Porod:2014xia}.
Unfortunately, for now our FeynRules output does not work properly
with FeynArts, FormCalc. 
\begin{table}[H]
\noindent \begin{centering}
\begin{tabular}{|c|c|c|}
\hline 
Process & Current Limit & Planned Limit\tabularnewline
\hline 
\hline 
$\tau\rightarrow\mu\gamma$ & 6.8E-8 & 1.0E-9\tabularnewline
\hline 
$\tau\rightarrow\mu\mu\mu$ & 3.2E-8 & 1.0E-9\tabularnewline
\hline 
$\tau\rightarrow eee$ & 3.6E-8 & 1.0E-9\tabularnewline
\hline 
$\mu\rightarrow e\gamma$ & 5.7E-13 & 1.0E-14\tabularnewline
\hline 
$\mu N\rightarrow eN$ & 7.0E-13 & 1.0E-17\tabularnewline
\hline 
$\mu\rightarrow eee$ & 1.0E-12 & 1.0E-16\tabularnewline
\hline 
\end{tabular}
\par\end{centering}

\protect\caption{Current and planned limits on Lepton CFLV}
\end{table}


\bibliographystyle{plain}
\bibliography{/Users/robert/Dropbox/Diary/Bibliography/base}

\end{document}
